\documentclass[12pt]{article}
\usepackage{amssymb}
\usepackage{multicol}
\usepackage{graphicx}
\usepackage{color}
\usepackage{amsthm}
\usepackage{hyperref}
\usepackage{amsmath}
\usepackage{verbatim}
\usepackage{caption}
\usepackage{datetime}
\usepackage{subcaption}
\usepackage{amsmath, epsfig}
\usepackage[latin1]{inputenc}
\usepackage{enumitem}
\newtheorem{theorem}{Theorem}
\newtheorem{lemma}{Lemma}
\newtheorem{corollary}{Corollary}
\newcommand{\abs}[1]{\left| #1\right|}
\newcommand{\Lap}[1]{\mathcal{L}\left\{#1\right\}}
\newcommand{\solution}[1]{
\color{red}\begin{quote}Solution:\quad 
\color{black} #1
\end{quote}\color{black}
}
\newcommand{\ba}{\backslash}
\newcommand{\Ber}{\hbox{Ber}}
\DeclareMathOperator{\Gcd}{gcd}
\renewcommand{\gcd}[2]{\Gcd\left(#1, #2\right)}
\newcommand{\p}[1]{\mathbb{P}\left(#1\right)}
\newcommand{\e}[1]{\mathbb{E}(#1)}
\newcommand{\Po}[1]{\hbox{Po}(#1)}
\newcommand{\var}[1]{\hbox{Var}(#1)}
\newcommand{\Z}{\mathbb{Z}}
\newcommand{\R}{\mathbb{R}}
\newcommand{\Q}{\mathbb{Q}}
\newcommand{\N}{\mathbb{N}}
\newcommand{\floor}[1]{\left\lfloor #1\right\rfloor}
\newcommand{\C}{\mathbb{C}}
\DeclareMathOperator{\Diam}{diam}
\newcommand{\diam}[1]{\Diam\left(#1\right)}
\renewcommand\qedsymbol{$\blacksquare$}

\newcommand{\Mod}[1]{\ (\bmod\ #1)}

\begin{document}
\title{21-127 Homework 10
}
\author{Christian Broms \\ Section J}
\date{\today}
\maketitle
Complete the following problems. Fully justify each response.


\begin{enumerate}

\item Let $X$ be a set, and let $\sim_1$ and $\sim_2$ be two equivalence relations on $X$. Define a relation $\sim$ by $x\sim y$ if and only if $x\sim_1 y \wedge x\sim_2 y$. Prove that this is an equivalence relation. Describe the equivalence classes $[x]_\sim$ in terms of $[x]_{\sim_1}$ and $[x]_{\sim_2}$.

\begin{proof}
Since $\sim_1$ and $\sim_2$ are both defined to be equivalence relations on $X$, we know that they are reflexive, symmetric, and transitive. We shall prove that $\sim$, defined by $x\sim y$ if and only if $x\sim_1 y \wedge x\sim_2 y$ also fufills these conditions. 

Reflexivity: Suppose $x \in [x]_\sim$, so $x \in [x]_{\sim_1}$ and $x \in [x]_{\sim_2}$. Then $x\sim_1 x$ and $x\sim_2 x$. Thus, $x\sim_1 x \wedge x\sim_2 x$, which implies $x\sim x$, so it is reflexive. 

Symmetry: Suppose $y \in [x]_\sim$. Then $y \in [x]_{\sim_1}$ and $y \in [x]_{\sim_2}$. So $y \sim_1 x$ and $y \sim_2 x$, and thus $y \sim x$. Since both $\sim_1$ and $\sim_2$ are equivalence relations, then both are symmetric, and $x \sim_1 y$ and $x \sim_2 y$. So, $x \sim y$. Thus, since $y \sim x$ and $x \sim y$, $\sim$ is symmetric. 

Transitivity: Suppose $x \in [y]_\sim$ and $y \in [z]_\sim$. Using the first assumption, we see that $x \in [y]_{\sim_1}$ and $x \in [y]_{\sim_1}$, so $x \sim_1 y$ and $x \sim_2 y$, and thus $x \sim y$ Using the second, we use the same logic to see that $y \sim_1 z$ and $y \sim_2 z$, so $y \sim z$. Additionally, since both $\sim_1$ and $\sim_2$ are equivalence relations and transitive, then $x \sim_1 z$ and $x \sim_2 z$. So, it follows that $x \sim z$. 

Since we have shown that $\sim$ is reflexive, symmetric, and transative, we conclude that it is an equivalence relation. 

The equivalence class $[x]_\sim$ can be described in terms of $[x]_{\sim_1}$ and $[x]_{\sim_2}$ by  $[x]_\sim = [x]_{\sim_1} \cap [x]_{\sim_2}$
\end{proof}

\item Let $P$ be a poset with element set $\N$, and order $\preceq$, where $a\preceq b$ if and only if $a|b$. Formally prove that this is a poset.

\begin{proof}
To prove that the set $P$, defined on $(\N,\ |\ )$ is a poset, we will show that it fufills the three requirements; reflexivity, antisymmetry, and transitivity.

Reflexivity: Let $x \in \N$. Then clearly, $x|x$ by definition of divisibility. 

Antisymmetry: Let $x, y \in \N$. Then we can write $x = yq$ and $y = xp$ for some $p, q \in \Z$. So $x = xpq$, and it must be that $pq = 1$, but since $p, q$ are integers, both $p$ and $q$ must be one. Thus, we conclude that $x = y$. 

Transitivity: Let $x, y, z \in \N$. Suppose $x | y$ and $y|z$. Again, we can write $x=yq$ and $y = zp$ for some $p, q \in \Z$. So $x = z(pq)$, and thus $x|z$. 


Having shown all three conditions, we conclude that $P$ must be a poset. 
\end{proof}

\item Let $P$ be a poset with partial order $\leq$. We define an element $m$ of $P$ to be {\it minimal} if there does not exist any $x\in P$ with $x<m$.
\begin{enumerate}
\item How is a minimal element different from a minimum element (as defined in class)?

A mimimum element is an element in a poset that is less than all other elements in the set, that is, for some $m\in X$, $m \leq x\ \forall x \in X$. On the other hand, a mimimal element is the lowest element in a subset of the set, so that it is less than all elements it can be compared to. The difference is that the minimal element does not have to be comparable to all elements in a poset, while being the smallest element. 

\item Give an example of a poset that has at least one minimal element, but no minimum element.

Consider the poset defined by $(X,|)$, where $X = \{2, 4, 6, 8, 5\}$. Notice that there is no minimum element in $X$, but there is at least one minimal element, namely 2 and 7.  

\item Prove that if a poset has a minimum element, then it has only one minimal element.

\begin{proof}
Without loss of generality, assume that a poset $P$ has two minimal elements, $a, b$, and one minimum element, $\perp$. By definition of minimum, for all $x \in P$, $\perp \leq x$. Therefore, $\perp \leq a$ and $\perp \leq b$. But by definition of a minimal element, for all $x \in S$ where $S \subseteq P$, $a \leq x$. However, since $\perp \leq x$ for all $x \in P$, it must be that $a = \perp$. The same argument can be applied to $b$, so we have $b = \perp$. Thus, $a = \perp = b$, which means that there can only be one minimal element. 
\end{proof}

\end{enumerate}
(Note: This entire problem could be repeated verbatim for a maximal element as well)

\item Let $X$ be a set, and let $P$ be the poset whose elements are $\mathcal{P}(X)$, and $A\leq B$ if and only if $A\subseteq B$.

Let $\{A_1, A_2, \dots, A_n\}$ be elements of $P$. Prove that $\displaystyle \bigwedge_{i=1}^n A_i = \displaystyle\bigcap_{i=1}^n A_i$, and $\displaystyle \bigvee_{i=1}^n A_i = \displaystyle\bigcup_{i=1}^n A_i$.

\begin{proof}
First, we will show that $\displaystyle \bigvee_{i=1}^n A_i = \displaystyle\bigcup_{i=1}^n A_i$ with induction on $n$. 

Base Case: $n = 1$. We have $\displaystyle \bigvee_{i=1}^1 A_i = \displaystyle \bigvee_{i=1}^1 A_1$. This is $A_1$ since $A_1 \subseteq A_1$. Additionally, note that $ \displaystyle\bigcup_{i=1}^1 A_i = A_1$, so $\displaystyle \bigvee_{i=1}^1 A_i = \displaystyle\bigcup_{i=1}^1 A_i$. 

Induction Step: Assume $\displaystyle \bigvee_{i=1}^{n} A_i = \displaystyle\bigcup_{i=1}^n A_i$. We want to show that $\displaystyle \bigvee_{i=1}^{n+1} A_i = \displaystyle\bigcup_{i=1}^{n+1} A_i$. We can rewrite $\displaystyle \bigvee_{i=1}^{n+1} A_i$ as $\displaystyle \bigvee_{i=1}^{n} A_i  \vee A_{n+1}$. We know that from the base case, $ \vee A_1 =  \cup A_1$. This applies to all single elements, so $ \vee A_{n+1} =  \cup A_{n+1}$, and $\displaystyle \bigvee_{i=1}^{n} A_i  \cup A_{n+1}$. Using the induction hypothesis, we can replace $\displaystyle \bigvee_{i=1}^{n} A_i$ to get $\displaystyle \bigcup_{i=1}^{n} A_i  \cup A_{n+1}$, so we have $\displaystyle \bigcup_{i =1}^{n+1} A_i = \displaystyle \bigvee_{i =1}^{n+1} A_i$. 

We can therefore conclude that $\displaystyle \bigvee_{i=1}^n A_i = \displaystyle\bigcup_{i=1}^n A_i$. The entire argument can be re-written with $\cap$ and $\wedge$, using the same logic above, so we have $\displaystyle \bigwedge_{i=1}^n A_i = \displaystyle\bigcap_{i=1}^n A_i$. 
\end{proof}

\item Let $(X, \preceq)$ be a lattice, and let $x, y\in X$. Prove the following:
\begin{enumerate}
\item $x\vee y = y \vee x$

\begin{proof}
Let $L_1 = \{i \in X\ |\ i \preceq x \vee y\}$ and $L_2 = \{i \in X\ |\ i \preceq y \vee x\}$. We will show that $L_1 = L_2$ by double containment. 

$(\subseteq)$. Let $i \in L_1$. Then $i \preceq x \vee y$, so $i \preceq x$ or $i \preceq y$. Thus, $i \preceq y \vee  x$, and therefore $i \in L_2$. 

$(\supseteq)$ Let $i \in L_2$. Then $i \preceq y \vee x$, so $i \preceq y$ or $i \preceq x$. Thus, $i \preceq x \vee y$, and therefore $i \in L_1$.

Thus, we conclude $L_1 = L_2$, and $x\vee y = y \vee x$. 
\end{proof}

\item $x\wedge y = y\wedge x$

\begin{proof}
Let $L_1 = \{i \in X\ |\ i \preceq x \wedge y\}$ and $L_2 = \{i \in X\ |\ i \preceq y \wedge x\}$. We will show that $L_1 = L_2$ by double containment. 

$(\subseteq)$. Let $i \in L_1$. Then $i \preceq x \wedge y$, so $i \preceq x$ and $i \preceq y$ . Thus, $i \preceq y \wedge  x$, and therefore $i \in L_2$. 

$(\supseteq)$ Let $i \in L_2$. Then $i \preceq y \wedge x$, so $i \preceq y$ and $i \preceq x$. Thus, $i \preceq x \wedge y$, and therefore $i \in L_1$.

Thus, we conclude $L_1 = L_2$, and $x\wedge y = y \wedge x$. 
\end{proof}
\item $x\vee (x\wedge y)=x$

\begin{proof}
Let $L_1 = \{i \in X\ |\ i \preceq x$ or $ i \preceq x \wedge y\}$ and $L_2 = \{i \in X\ |\ i \preceq x\}$. We will show that $L_1 = L_2$ by double containment. 

$(\subseteq)$. Let $i \in L_1$. Then $i \preceq x$ or $ i \preceq x \wedge y$. In both cases, it is true that $i \preceq x$, so $i \in L_2$.

$(\supseteq)$. Let $i \in L_2$. Then $i \preceq x$. There are two scenarios. Either $i \preceq y$ pr $i \npreceq y$. In the first case, $i \preceq x$ and $i \preceq y$. In the second case, $i \preceq x$ and $i \npreceq y$. So, $i \preceq x$ or $i\preceq x$ and $i \preceq y$. Thus, $i \preceq L_1$. 

Thus, we conclude $L_1 = L_2$, and $x\vee (x\wedge y)=x$. 
\end{proof}

\item $x\wedge(x\vee y)=x$

\begin{proof}
Let $L_1 = \{i \in X\ |\ i \preceq x$ and $ i \preceq x \vee y\}$ and $L_2 = \{i \in X\ |\ i \preceq x\}$. We will show that $L_1 = L_2$ by double containment. 

$(\subseteq)$. Let $i \in L_1$. Then $i \preceq x$ and $ i \preceq x \vee y$. In both cases, it is true that $i \preceq x$, so $i \in L_2$.

$(\supseteq)$. Let $i \in L_2$. Then $i \preceq x$. Then, $x \preceq x \vee y$ by definiton of a poset. So by transitivity, it follows that if $i \preceq x$ and $i \preceq x \vee y$,  then $i \preceq x \wedge (x \vee y)$, and $i \in L_1$. 

Thus, we conclude $L_1 = L_2$, and $x\wedge(x\vee y)=x$. 
\end{proof}
\end{enumerate}

\item Give an example of a lattice that is not distributive.

Consider the lattice defined by $X = \{a, b, c, d, e\}$, $(X, \leq)$, where $\leq\ = \{(d,a),(d, b), (d, c), (a, e), (b, e), (c, e)\}$. (this is the diamond shaped lattice in a hasse diagram) Notice that the distributive lattice property of distributivity does not hold: $(a \wedge b) \vee (b \wedge c) = d \vee d = d$. On the other hand, $a \wedge (b \vee c) = a \wedge e = a$. The distributive property does not hold, as $a \neq d$. 


\end{enumerate}
\end{document}