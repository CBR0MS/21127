\documentclass{article}
\usepackage{amssymb}
\usepackage{multicol}
\usepackage{graphicx}
\usepackage{color}
\usepackage{amsthm}
\usepackage{hyperref}
\usepackage{amsmath}
\usepackage{verbatim}
\usepackage{caption}
\usepackage{subcaption}
\usepackage{amsmath, epsfig}
\setlength{\parskip}{8pt}

\newtheorem{theorem}{Theorem}
\newtheorem{lemma}{Lemma}
\newtheorem{corollary}{Corollary}
\theoremstyle{definition}
\newtheorem{definition}{Definition}
\newtheorem{exercise}{Exercise}
\renewcommand{\today}{}
\newcommand{\abs}[1]{\left| #1\right|}
\newcommand{\Lap}[1]{\mathcal{L}\left\{#1\right\}}


\addtolength{\topmargin}{-.875in}
\addtolength{\textheight}{1.75in}

\newcommand{\solution}[1]{
\color{red}\begin{quote}Solution:\quad 
\color{black} #1
\end{quote}\color{black}
}
\newcommand{\ba}{\backslash}
\DeclareMathOperator{\Gcd}{gcd}
\renewcommand{\gcd}[2]{\Gcd\left(#1, #2\right)}
\newcommand{\Z}{\mathbb{Z}}
\newcommand{\R}{\mathbb{R}}
\newcommand{\Q}{\mathbb{Q}}
\newcommand{\N}{\mathbb{N}}
\newcommand{\floor}[1]{\left\lfloor #1\right\rfloor}
\newcommand{\C}{\mathbb{C}}
\DeclareMathOperator{\Diam}{diam}
\newcommand{\Mod}[1]{\ (\bmod\ #1)}
\newcommand{\diam}[1]{\Diam\left(#1\right)}


\begin{document}
\title{21-127 Bonus}
\author{Christian Broms}

\maketitle


\begin{exercise}
Prove that if $A$ is a cut, and $r\notin A$, then $p<r$ for every $p\in A$.
\end{exercise}

\begin{proof}
By the first property of a cut, we have that for $x, y \in \Q$, if $y \leq x$, then $y \in A$, where $A \subseteq \Q$. If $r \notin A$, $p \in A$, then $r \nleq p$ because this would contradict the defintion of a cut. Thus, it must be that $r > p$, and therefore we have that $p < r$ for every $p \in A$. 
\end{proof}


\begin{exercise}\label{ratcuts}
Prove that for any $q\in \Q$, there exists a cut $A_q$ for which $q$ is a least upper bound.
\end{exercise}

\begin{proof}
Assume for some $q \in \Q$, $A_q \subseteq \Q$. By the second property of a cut, $A_q$ has an upper bound at $q$. Thus, we have that $q \geq a\ \forall a \in A_q$. Additionally, if $q' \geq a\ \forall a \in A_q$, then $q' \geq q$. Therefore the properties of a least upper bound are fufilled and we can conclude that for any $q\in \Q$, there exists a cut $A_q$ for which $q$ is a least upper bound.
\end{proof}

\begin{exercise}\label{ratorder}
Prove that if $p, q\in \Q$, with $p< q$, then $A_p\subset A_q$.
\end{exercise}

\begin{proof}
Let $x \in A_p$. By Exercise 2 we know that $x < p$. Since $p < q$, then $x < q$ due to the transitive property of $\Q$, an ordered field. If $x < q$, and $q$ is the least upper bound of $A_q$, then $x \in A_q$ by definition of least upper bound. Thus, we have that $A_p\subset A_q$. 
\end{proof}

\begin{exercise}\label{sqrtcut}
Prove that $\{x\in \Q\ | \ x^2\leq 2\hbox{ or } x<0\}$ is a cut.
\end{exercise}

\begin{proof}
 Set $A = \{x\in \Q\ | \ x^2\leq 2\hbox{ or } x<0\}$. We wish to show that $A$ fufills the properties of a cut. There are two cases, either $x < 0$ or $x^2\leq 2$. 

Case 1: $x < 0$. 
\begin{enumerate}
\item Suppose $x, y \in \Q$,  $x \in A$, and $y \leq x$.  Since $x \in A$, $x < 0$, and since $y \leq x$, it must be that $y < 0$ as well, by transitivity. 
\item The upper bound of $x < 0$ is 0, since $\forall x \in A$, where $x < 0$, $x \ngeq 0$. Thus, 0 is an upper bound.

\item (Do this case) 
\end{enumerate}

Case 2: $x^2\leq 2$
\begin{enumerate}
\item (Do this case) 
\item (Do this case) 
\item (Do this case) 
\end{enumerate}

\end{proof}

\begin{exercise}
Prove that the ordering defined above is a total order; that is, it is a partial order, and any two elements of $\mathcal{C}$ are comparable.
\end{exercise}

\begin{exercise}
Prove the following are true about this definition of addition:
\begin{enumerate}
\item Closure: For all $A, B\in \mathcal{C}$, $A+B$ is a cut (so that when we add cuts, we stay within the universe of cuts).
\item Associativity: $A+(B+C)=(A+B)+C$ for all $A, B, C\in \mathcal{C}$
\item Identity: $A+A_0=A$ for all $A\in \mathcal{C}$
\item Existence of inverses: For all $A\in \mathcal{C}$ there exists $B\in \mathcal{C}$ such that $A+B=A_0$. (Hint: set $B=\{p\in \Q\ | \ \exists r\notin A\hbox{ such that }p<-r\}$. Then prove that $B$ is a cut, and $A+B=A_0$.)
\item Commutativity: For all $A, B\in \mathcal{C}$, $A+B=B+A$.
\item Order arithmetic: For all $A, B, C\in \mathcal{C}$, if $A\leq B$, then $A+C\leq B+C$.
\end{enumerate}
\end{exercise}

\begin{proof}
\begin{enumerate}

\item Closure: We will prove that $A + B$ fufills the properties of a cut. 
\begin{enumerate}
\item We will prove if $x, y \in \Q$, $x \in A+B$, and $ y \leq x$, then $y \in A + B$. There are two cases, either $y < x$ or $y = x$. 

Case 1: $y < x$. Let $y = a + b$, such that $a \in A$ with $a < p$, and $b \in B$, with $b < q$. We have the property that $a + b < p + q$ since $a < p$ and $b < q$, as proven in a previous homework. Since $A$ and $B$ are cuts, and $A$ contains all elements less than $p$ and $B$ has all elements less than $q$, then $a + b$ represents all elements less than $p + q$. So, we have that $y \in A + B$. 

Case 2: $y = x$. Let $y = p + q$. Clearly, it follows that $y \in A + B$. 

Thus, in both cases, we get $y \in A + B$. 

\item Since $A$ and $B$ are cuts, they both have upper bounds. Let the upper bounds be denoted $a$ and $b$, respectively. Let $p + q \in A + B$ such that $p \in A$ and $q \in B$. So, $ p \leq a$ and $q \leq b$, so we have the property that $p + q \leq a + b$, as proven previously. Thus, by defintion, $a + b$ is an upper bound. 

\item Let $x = a + b$, with $a \in A$ and $b \in B$. Then $\forall a \in A$, $\exists p \in A$ such that $a < p$. Similarly, $\forall b \in B$, $\exists q \in A$ such that $b < q$. So, $ a+ b < p + q$, as previously proven. Thus, there exists some $y = p + q$ such that $\forall x \in A + B$, $\exists y \in A + B$ with $x < y$.  
\end{enumerate}
Thus, since $A+B$ fufills all three properties, it must be a cut. 

\item Associativity: Let $A + (B + C) = \{a + (b + c)\ |\ a \in A, b \in B , c \in C\}$ and $(A + B) + C = \{(a + b) + c\ |\ a \in A, b \in B , c \in C\}$. We have that $a + (b + c) = (a + b) + c$ by the axiom of associativity of addition. Suppose $p \in A + (B + C)$ and $q \in (A + B) + C$, so $ p = a + (b + c)$ and $q = (a + b) + c$. Thus, $p= q$. So, $ p \in (A + B) + C $ and $q \in A + (B + C)$. Since $p$ and $q$ are elements of both sets, it follows that $A + (B + C) \subseteq (A + B) + C$ and $(A + B) + C \subseteq A + (B + C) $, so $(A + B) + C = A + (B + C) $. 

\item Identity: Let $A + A_0 = \{p + q\ |\ p \in A, q \in A_0\}$. Suppose $q = 0$. Then $p + q = p + 0 = p$ by the first field axiom. Since $ p + q = p\ \forall p$ when $q = 0$, then $\forall p \in A$, $p \in A + A_0$. So, $ A \subseteq A + A_0$. To prove that $A + A_0 \subseteq A$, there are two cases, $q = 0$ and $q < 0$. 

Case 1: Let $q < 0$. Then , $p + q < p$. So, $p + q \in A$. 


Case 2: Let $q = 0$.Then $p + q \in A + A_0$ and $p + q = p + 0 = p$, so $p + q \in A$. 

Thus, $A + A_0 \subseteq A$. Since we have shown both sides of containment, we conclude that $ A = A+ A_0$. 
\item Existence of Inverses: (prove this case)
\item Communitivity: Let $A + B = \{ a + b\ |\ a \in A, b \in B\}$ and $B + A = \{ b + a\ |\ a \in A, b \in B\}$. So, we know that $\forall a + b \in A + B, \exists a + b \in B + A$, since $a + b = b + a$ by communitivity of addition. So, $A + B \subseteq B + A$. Using the same logic, we get $B + A \subseteq A + B$, so we conclude that $ A + B = B + A$. 
\item Order Arithmatic: (prove this case).
\end{enumerate}
\end{proof}

\begin{exercise}
Let $p\in \Q$. Prove that $-A_p=A_{-p}$.
\end{exercise}

\begin{exercise}
Prove that for $A, B> A_0$, the definition of $A\cdot B$ above yields a cut; that is, $A\cdot B\in \mathcal{C}$.
\end{exercise}

\begin{exercise}
Prove that if $p, q\in \Q$, and $p, q> 0$, then $A_p\cdot A_q = A_{pq}$.
\end{exercise}

\begin{exercise}\label{negmove}
For all $A, B\in \mathcal{C}$, prove that $-(A\cdot B) = (-A)\cdot B=A\cdot(-B)$.

(Note: You will need to consider cases, depending on the sign of $A$ and $B$.)
\end{exercise}

\begin{exercise}Prove the following are true about this definition of multiplication:
\begin{enumerate}
\item Identity: $A\cdot A_1 = A$ for all $A\in \mathcal{C}$.
\item Inverse: For all $A\in \mathcal{C}$ with $A\neq A_0$, there exists $B\in \mathcal{C}$ such that $A\cdot B=A_1$. (Hint: set $B=\{p\in \Q\ | \ \exists r\notin A, p<\frac {1}{r}\}$ when $A>0$.)
\item Commutativity: $A\cdot B=B\cdot A$ for all $A, B\in \mathcal{C}$.
\item Distributivity: $A\cdot(B+C) = A\cdot B+A\cdot C$ for all $A, B, C\in \mathcal{C}$.
\item Order arithmetic: For all $A, B\in \mathcal{C}$, if $A, B\geq A_0$ then $A\cdot B\geq A_0$.
\end{enumerate}
\end{exercise}

\begin{exercise}
Let $\mathcal{A}\subseteq \mathcal{C}$ be a set of cuts. Prove that there exists $S\in \mathcal{C}$ that satisfies the definition of a supremum.
\end{exercise}

\end{document}