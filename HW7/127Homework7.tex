\documentclass[12pt]{article}
\usepackage{amssymb}
\usepackage{multicol}
\usepackage{graphicx}
\usepackage{color}
\usepackage{amsthm}
\usepackage{hyperref}
\usepackage{amsmath}
\usepackage{verbatim}
\usepackage{caption}
\usepackage{datetime}
\usepackage{subcaption}
\usepackage{amsmath, epsfig}
\usepackage[latin1]{inputenc}
\usepackage{enumitem}
\newtheorem{theorem}{Theorem}
\newtheorem{lemma}{Lemma}
\newtheorem{corollary}{Corollary}
\newcommand{\abs}[1]{\left| #1\right|}
\newcommand{\Lap}[1]{\mathcal{L}\left\{#1\right\}}
\newcommand{\solution}[1]{
\color{red}\begin{quote}Solution:\quad 
\color{black} #1
\end{quote}\color{black}
}
\newcommand{\ba}{\backslash}
\newcommand{\Ber}{\hbox{Ber}}
\DeclareMathOperator{\Gcd}{gcd}
\renewcommand{\gcd}[2]{\Gcd\left(#1, #2\right)}
\newcommand{\p}[1]{\mathbb{P}\left(#1\right)}
\newcommand{\e}[1]{\mathbb{E}(#1)}
\newcommand{\Po}[1]{\hbox{Po}(#1)}
\newcommand{\var}[1]{\hbox{Var}(#1)}
\newcommand{\Z}{\mathbb{Z}}
\newcommand{\R}{\mathbb{R}}
\newcommand{\Q}{\mathbb{Q}}
\newcommand{\N}{\mathbb{N}}
\newcommand{\floor}[1]{\left\lfloor #1\right\rfloor}
\newcommand{\C}{\mathbb{C}}
\DeclareMathOperator{\Diam}{diam}
\newcommand{\diam}[1]{\Diam\left(#1\right)}
\renewcommand\qedsymbol{$\blacksquare$}

\newcommand{\Mod}[1]{\ (\bmod\ #1)}


\begin{document}
\title{21-127 Homework 7}
\author{Christian Broms \\ Section J}
\date{\today}
\maketitle

Complete the following problems. Fully justify each response.
\begin{enumerate}

\item Let $a,b\in \Z$, $n\in \N$ with $n\geq 1$, and $k, \ell\in\N$ with $k\equiv \ell\Mod{n}$ and $a\equiv b\Mod{n}$.
\begin{enumerate}
\item Is it true that $a^k\equiv b^k\Mod{n}$? If so, prove it. If not, provide a counterexample.

Yes. 
\begin{proof}
We begin by proving that $a \equiv b \Mod{n}$ and $c \equiv d \Mod{n}$ implies $ac \equiv bd \Mod{n}$. We can write $a = kn + b$ and $c = k'n + d$ for some $k, k' \in \Z$. So $ac = k'kn^2 + (bk' + dk)n + bd$. Because this is in mod $n$, we can eliminate all factors of $n$ so $ac \equiv bd \Mod{n}$. 

Next, we will proceed with induction to show $a^k\equiv b^k\Mod{n}$. 

Base Case: $k = 1$, so $a^1 = b^1 \Mod{n}$ is true by the given information. 

Induction Hypothesis: Assume $a^k\equiv b^k\Mod{n}$. We will show $a^{k+1}\equiv b^{k+1}\Mod{n}$. We know $a\equiv b\Mod{n}$ and $a^k\equiv b^k\Mod{n}$ by IH, so we multiply these together using the proven fact above to get $a^{k+1}\equiv b^{k+1}\Mod{n}$, and the induction holds. 

Thus, we can safely say that $a^k\equiv b^k\Mod{n}$. 
\end{proof}

\item Is it true that $a^k\equiv a^\ell\Mod{n}$? If so, prove it. if not, provide a counterexample.

False. Consider the following counterexample: $a = 3, n = 5, k = 7, \ell =2 $. So $3^7 \not\equiv 3^2 \Mod{5}$
\end{enumerate}

\item Let $a\in\Z$, and let $n\in\N$ with $n\geq 1$. Suppose that $a\perp n$. Show that $u, u'$ are both multiplicative inverses for $a$ if and only if $u$ is a multiplicative inverse for $a$ and $u\equiv u'\Mod{n}$.

\begin{proof}
($\Rightarrow$) Assume $u, u'$ are multiplicative inverses for $a$, such that $au \equiv 1 \Mod{n}$ and $au' \equiv 1 \Mod{n}$. Then, since $a$ and $n$ are coprime, we manipulate then divide by $a$ so $a(u - u') \equiv 0 \Mod{n}$ and $u - u' \equiv 0 \Mod{n}$, so $u \equiv u' \Mod{n}$ and we are done. 

($\Leftarrow$) Assume $u$ is a multiplicative inverse for $a$ and $u\equiv u'\Mod{n}$, so we have $au \equiv 1 \Mod{n}$ and $u \equiv u' \Mod{n}$. We can multiply by $a$ to get $au \equiv au' \Mod{n}$. Since we know $au \equiv 1 \Mod{n}$ and $au \equiv au' \Mod{n}$, we can say $au' \equiv 1 \Mod{n}$. This, taken with the fact $au \equiv 1 \Mod{n}$ implies that $u, u'$ are multiplicative inverses for $a$. 

Therefore, since we have shown both sides of implication, it follows that $u, u'$ are both multiplicative inverses for $a$ if and only if $u$ is a multiplicative inverse for $a$ and $u\equiv u'\Mod{n}$.
\end{proof}

\item Let $p$ be a positive prime, and $k\in \N$ with $k\geq 1$. Prove that $\varphi(p^k) = p^k-p^{k-1}$.

\begin{proof}
The totient function is defined as $\varphi(n) = $ the number of integers between 1 and $n$ that are coprime to $n$. In this case, we are looking for some $m \in \Z$ such that $\gcd{m}{p^k} = 1$, so we need some $m$ that does not divide $p^k$. We list the number of integers between 1 and $p^k$ that are divisible by $p$ as $1p, 2p, 3p \dots p^{k-1}p$. So there are $p^{k-1}$ such numbers, and our set $\{1, 2, 3, \dots p^k\}$ has $p^k - p^{k-1}$ numbers that are not divisible by $p^k$. Therefore, we conclude $\varphi(p^k) = p^k-p^{k-1}$.
\end{proof}

\item Read the proof of Theorem 3.3.49 and Example 3.3.51. Then prove that for any $b\in \N$ with $b\geq 2$, and $a\in \N$, $a$ is divisible by $b-1$ if and only if the sum of the base $b$ digits of $a$ is divisible by $b-1$.

\begin{proof}
We can write $a$ in its base $b$ expansion as $a = d_r d_{r-1} \dots d_1 d_0$ base $b$, such that $ a = \sum\limits_{i=0}^r d_ib^i$. So the sum of these digits can be written as $s = \sum\limits_{i=0}^r d_i$. Because we defined $a$ as $\sum\limits_{i=0}^r d_ib^i$, we can say $s \equiv \sum\limits_{i=0}^r d_ib^i \Mod{b-1}$. We can further reduce this $s \equiv \sum\limits_{i=0}^r d_i1^i \Mod{b-1}$ because $b \equiv 1 \Mod{b-1}$. Finally, we can reduce to $s \equiv \sum\limits_{i=0}^r d_i \Mod{b-1}$ because $1^i$ is always 1. So $s \equiv a \Mod{b-1}$. It therefore follows by definition of congruence that $a$ is divisible by $b-1$ if and only if the sum of the base $b$ digits of $a$ is divisible by $b-1$.
\end{proof}

\item For each of the following functions, determine if it is injective, surjective, both, or neither. Prove that your answers are correct.

\begin{enumerate}
\item $f:\Z\to\N$, $f(x) = x^2$.

Injective: No. Consider $f(2) = f(-2)$, but $-2 \neq 2$. 

Surjective: No. Consider $x = 3$, but there is no $z \in \Z$ such that $z^2 = 3$, as $\ \sqrt[]{3} \notin \Z$
\item $g:\N\to\Z$, $g(x)=x^2$.

Injective: Yes. Assume $f(x) = f(y)$, so $x^2 = y^2$ and $x = y$. 

Surjective: No. Consider $x = 3$, but there is no $n \in \N$ such that $n^2 = 3$, as $\ \sqrt[]{3} \notin \N$
\item $h:\R\to\Z$, $h(x) = \lfloor x\rfloor$ 

(note: $\lfloor x\rfloor$ is the number you get by rounding $x$ down to the nearest integer. Formally, we define \[\lfloor x\rfloor = \max\{ y\in \Z\ | \ y\leq x\}.\] You may be reasonably skeptical that such a number exists, since we cannot apply the Well-Ordering Principle here.... so if you are skeptical, prove it.)


Injective: No. Consider $h(\pi) = h(3)$, but $\pi \neq 3$. 

Surjective: Yes. Let $y \in Z$. Let $x = y + \frac{1}{2}$. then $y < x < y + 1$, so $\max\{ n\in \Z\ | \ n\leq x\} = y$. Thus, $f(x) = \lfloor x\rfloor = y$


\item $f: \N\to \Z$, $f(x) = \left\{\begin{array}{cl} \frac{x}{2} & x\hbox{ is even}\\ -\frac{x+1}{2} & x\hbox{ is odd}\end{array}\right.$

Injective: Yes. Suppose $f(x) = f(y)$. So $f(x)$ is either even or odd. In the first case, take $f(x)$ to be odd. Then $f(x) = -\frac{x+1}{2}$ and $f(y) = -\frac{y+1}{2}$, thus $-\frac{x+1}{2} = -\frac{y+1}{2}$, so clearly $x = y$. In the second case, take $f(x)$ to be even. Then $f(x) = \frac{x}{2}$ and $f(y) = \frac{y}{2}$, thus $\frac{x}{2} = \frac{y}{2}$, and therefore $x = y$.

Surjective: Let $y \in \Z$. If $y$ is odd, then $x = -2y - 1$, so $f(-2y - 1) = -\frac{x+1}{2}$ and thus $f(x) = y$. If $y$ is even, then $x = 2y$, so $f(2y) = \frac{y}{2}$, and clearly $f(x) = y$. So the function is surjective.

\end{enumerate}

\item Let $f:X\to Y$ and $g:Y\to Z$ be bijective functions. Prove that $g\circ f$ is also bijective. Is the converse true?

\begin{proof}
Since $f$ and $g$ are bijective, then by definiton they are also surjective. So, for some $x \in X$, $y \in Y$, $z \in Z$, we can say $f(x) = y$, $g(y) = z$. So, we know $z = g(y) = g(f(x)) = g \circ f(x) = g \circ f$. Hence, $g \circ f$ is surjective. 

Next, since we know $f$ and $g$ are bijective, then by definiton they are also injective. So, for some $x, y \in X$ we say $g \circ f(x) = g \circ f(y)$. Then, $g(f(x)) = g(f(y))$. Since $g$ is injective, this implies $f(x) = f(y)$. Moreover, $f$ is injective, so $x = y$. Hence, $g \circ f$ is injective. 

Thus, we can say that if $g, f$ are bijective, then $g\circ f$ is also bijective. 

The converse is not true. The converse is let $g \circ f$ be bijective, then $g$ and $f$ are both bijective. We can show this is not true by constructing two functions, $g, f$ that are not bijective, but $g \circ f$ is bijective. If we take $g: \Z \times \Z \rightarrow \Z$, $g(x, y) = (x)$ and $f: \Z \rightarrow \Z \times \Z$ and  $f(x) = (x, 0) $, then $g \circ f$ is bijective, while $g$ is not bijective, because it is not injective, and $f$ also not bijective, because it is not surjective. 
\end{proof}


\end{enumerate}


\end{document}